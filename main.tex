% Sergio Miguens Iglesias' CV. This document is based on a template, see LICENSE
% for more details. You are free to copy and redistribute the code for my CV
% freely.

%                                      PACKAGES AND OTHER DOCUMENT CONFIGURATIONS
%%%%%%%%%%%%%%%%%%%%%%%%%%%%%%%%%%%%%%%%%%%%%%%%%%%%%%%%%%%%%%%%%%%%%%%%%%%%%%%%%

\documentclass[9pt]{developercv} % Default font size, values from 8-12pt are recommended
\usepackage{pagecolor,lipsum}
\usepackage{amssymb}
\usepackage{tikz}

% Cool little trick to have two languages in the same file.
\newif\ifen
\newif\ifes
\newcommand{\en}[1]{\ifen#1\fi}
\newcommand{\es}[1]{\ifes#1\fi}

\ifdefined\eslang
  \estrue
\else
  \entrue
\fi

\begin{document}

%                                                   TITLE AND CONTACT INFORMATION
%%%%%%%%%%%%%%%%%%%%%%%%%%%%%%%%%%%%%%%%%%%%%%%%%%%%%%%%%%%%%%%%%%%%%%%%%%%%%%%%%

% Name and title.
\begin{minipage}[t]{0.45\textwidth}
    \vspace{-\baselineskip}

    \colorbox{black}{{\HUGE\textcolor{white}{\textbf{\MakeUppercase{Sergio}}}}}
    \colorbox{black}{{\HUGE\textcolor{white}{\textbf{\MakeUppercase{M. Iglesias}}}}}

    \vspace{6pt}

    {\huge
      \es{Ingeniero informático}
      \en{Computer Engineer}
    }
\end{minipage}

% Photo in header.
\begin{minipage}[t]{0.45\textwidth}
  \vspace{-\baselineskip}
  \vspace{-3.5cm}
  \hspace{12.5cm}
  \begin{tikzpicture}
    \clip (0,0) circle (2cm) node {\includegraphics[width=4cm]{img/me.jpg}};
  \end{tikzpicture}
\end{minipage}

\vspace{0.5cm}

%                                           INTRODUCTION, SKILLS AND TECHNOLOGIES
%%%%%%%%%%%%%%%%%%%%%%%%%%%%%%%%%%%%%%%%%%%%%%%%%%%%%%%%%%%%%%%%%%%%%%%%%%%%%%%%%

% About me.
\begin{minipage}[t]{0.35\textwidth}
  \vspace{-\baselineskip}
  \cvsect{
    \es{Sobre mí}
    \en{About me}
  }

  \es{
    Soy Sergio,
    un ingeniero informático de Vilagarcía de Arousa.
    Me apasionan los sistemas informáticos y la programación de bajo nivel.
    Soy usuario exclusivamente de Linux desde hace años.
  }
  \en{
    My name is Sergio,
	a computer engineer from Vilagarcía de Arousa, Spain.
	I am passionate about computer systems and low level programming.
  }
\end{minipage}
% Social icons: first column.
\begin{minipage}[t]{0.1\textwidth}
  \hspace{0.1cm}
\end{minipage}
\begin{minipage}[t]{0.275\textwidth}
  \vspace{1cm}
  \vspace{-\baselineskip}

  \icon{MapMarker}{12}{Pontevedra}\\
  \icon{Phone}{12}{+34 671307714}\\
  \icon{At}{12}{\href{mailto:sergio@lony.xyz}{sergio@lony.xyz}}\\
\end{minipage}
% Social icons: second column.
\begin{minipage}[t]{0.275\textwidth}
  \vspace{1cm}
  \vspace{-\baselineskip}

  \icon{Globe}{12}{\href{https://lony.xyz}{lony.xyz}}\\
  \icon{Github}{12}{\href{https://github.com/lonyelon}{github.com/lonyelon}}\\
\end{minipage}

\cvsect{
  \es{Habilidades}
  \en{Skills}
}

\begin{minipage}[t]{1.0\textwidth}
  \begin{center}
    {\Large
      \begin{tabular}{rl}
        $\bigstar\bigstar\bigstar\bigstar\bigstar$ & \hspace{0.25cm}
        \es{
          Programación de bajo nivel (C, C++, RUST),
          administración de sistemas,
        }
        \en{
          Low-level programming (C, C++, RUST),
          systems administration,
        }\\
        & \hspace{0.25cm}
        \es{
          trabajo en equipo,
          documentación,
        }
        \en{
          teamwork,
          documentation,
        }
        \ldots \vspace{0.25cm} \\
        $\bigstar\bigstar\bigstar$ & \hspace{0.25cm}
        \es{
          Programación de alto nivel (Java, JS, Python),
          bases de datos,
        }
        \en{
          High-level programming (Java, JS, Python),
          databases,
        }
        \ldots \vspace{0.25cm} \\
        $\bigstar$ & \hspace{0.25cm}
        \es{
          Diseño web,
          diseño de interfaces de usuario,
        }
        \en{
          Web design,
          user interface design,
        }
        \ldots   \vspace{0.25cm} \\
      \end{tabular}
    }
  \end{center}
\end{minipage}

%                                                                      EXPERIENCE
%%%%%%%%%%%%%%%%%%%%%%%%%%%%%%%%%%%%%%%%%%%%%%%%%%%%%%%%%%%%%%%%%%%%%%%%%%%%%%%%%

\cvsect{
  \es{Estudios y experiencia}
  \en{Experience and studies}
}

\begin{entrylist}

  % Civir
  \entry
    {2022 -- \es{actualidad}\en{now}}
    {
      \es{Desarrollador Openshift}
      \en{Openshift developer}
    }
    {Civir/Grupo Santander}
    {
      \es{
        Trabajando para la consultoría Civir, doy servicio al grupo Santander como desarrollador Openshift.
        Mi trabajo consiste en desarrollar los automatismos necesarios para los clústeres y asegurar que estos se ejecutan sin problemasen entornos muy diferentes.
      }
      \en{
        Working as a consultant for Civir, I develop Openshift tools for the Santander group.
        Mi work consists on creating the automations needed for clusters,
        as well as ensuring they run on many platforms without problems.
      }

      \texttt{Kubernetes/Openshift}\slashsep
      \texttt{Ansible}\slashsep
      \texttt{Scrum}\slashsep
      \texttt{Microsoft Azure}
    }

  % DXC
  \entry
    {2021 -- 2022}
    {
      \es{Ingeniero de sistemas backup}
      \en{Backup systems engineer}
    }
    {FDS -- a DXC Technology Company}
    {
      \es{
        Mi trabajo consistía en administrar las copias de seguridad de miles de máquinas.
        Dada la naturaleza del trabajo,
        tenía que interactuar a dirario con sistemas Windows y Linux,
        hacer scripts y vivir en el terminal.
      }
      \en{
        My work consisted on administering the backups for thousands of machines,
        which means I had to interact with Windows and Linux systems,
        write shell scripts and live in the terminal.
      }

      \texttt{RHLE}\slashsep
      \texttt{Veritas Netbackup}\slashsep
      \texttt{Docker}\slashsep
      \texttt{Python}\slashsep
      \texttt{Apache Airflow}
      \slashsep\texttt{CheckMK}
    }

  % University.
  \entry
      {2017 -- 2022}
      {
        \es{Grado en ingeniería informática}
        \en{Computer engineering degree}
      }
      {Escuela técnica superior de ingeniería, USC}
      {
        \es{
          Si algo he aprendido del grado es a trabajar con una gran carga de trabajo,
          deadlines inminentes y equipos diversos.

          Mi TFG fue un trabajo en equipo,
          en el que desarrollamos un entrenador personal virtual completo,
          incluyendo una página web,
          aplicación para Android,
          base de datos MySQL,
          CPD con Docker,
          y aplicación standalone para sistemas Windows.
          Este trabajo en equipo requirió de la aplicación de SCRUM y otras metodologías ágiles.
        }
        \en{
          The thing I learned in uni,
          over everything else,
          was to work under pressure,
          with incoming deadlines and diverse teams.

          My end-of-degree project was a team effort,
          where me and my colleagues designed a gym personal trainer,
          including a website,
          an Android app,
          a MySQL database,
          a CPD with Docker,
          and a standalone application for Windows systems.
        }
      }
\end{entrylist}

%                                                                 CERTIFICACIONES
%%%%%%%%%%%%%%%%%%%%%%%%%%%%%%%%%%%%%%%%%%%%%%%%%%%%%%%%%%%%%%%%%%%%%%%%%%%%%%%%%

\cvsect{
  \es{Certificaciones oficiales}
  \en{Official certifications}
}

\setlength{\tabcolsep}{12pt}
\begin{tabular}{ccc}
  \textbf{DevOps Foundation} & \textbf{VMWare vSphere 6.7 Foundations} & \textbf{ITIL4 Foundation} \\
  DevOps Institute & VMware & PeopleCert \\
  \includegraphics[height=3cm]{img/cert_devops.jpg} &
  \includegraphics[height=3cm]{img/cert_vmware-foundations.png} &
  \includegraphics[height=3cm]{img/cert_itil4-foundation.png}
\end{tabular}\\

%                                                                          EXTRAS
%%%%%%%%%%%%%%%%%%%%%%%%%%%%%%%%%%%%%%%%%%%%%%%%%%%%%%%%%%%%%%%%%%%%%%%%%%%%%%%%%

\begin{minipage}[t]{0.3\textwidth}
  \vspace{-\baselineskip}

  \cvsect{
    \es{Idiomas}
    \en{Languages}
  }

  \begin{tabular}{rl}
  \textbf{Español}   & Nativo \\
  \textbf{Galego}    & Nativo \\
  \textbf{English}   & Competent \\
  \textbf{Português} & Básico
  \end{tabular}
\end{minipage}
\hfill
\begin{minipage}[t]{0.3\textwidth}
  \vspace{-\baselineskip}

  \cvsect{Hobbies}

  \es{
    Me encanta leer y la halterofilia.
    Me muero por el software libre y la filosofía UNIX.
  }
  \en{
    I love reading and lifiting.
	I am really passionate about free software and the Unix philosophy.
	I just recently started picking up writing, but I'm not great yet!
  }
\end{minipage}
\hfill
\begin{minipage}[t]{0.3\textwidth}
  \vspace{-\baselineskip}

  \cvsect{
    \es{Contribuciones a software libre}
    \en{FOSS contributions}
  }
  \begin{barchart}{2.5}
    \baritem{Linux kernel}{40}
    \baritem{Logseq}{30}
    \baritem{Synapse (Matrix)}{10}
  \end{barchart}
\end{minipage}

\newpage

% TODO Complete this.
%\begin{minipage}[t]{0.9\textwidth}
%\cvsect{Proyectos personales}
%    \begin{entrylist}
%      \entry {2016 -- 2017} {CPU casera} {\href{https://www.reddit.com/r/computers/comments/65pxx3/8_bit_homebrew_computer/}{\underline{\textbf{ver en Reddit}}}} {Es una CPU de 8 bits que construí en bachillerato usando \textit{breadboards} y circuítos CMOS.}
%      \entry {2019} {Imitación de Minecraft en C++} {\href{https://github.com/lonyelon/Opencraft}{\underline{ver en Github}}} {Usa OpenGL y libnoise.}
%      \entry {2019} {Escáner de URLs} {\href{https://github.com/lonyelon/FuzzYourYelon}{\underline{ver en Github}}} {Se le proporciona una URL y la escanea en busca de ficheros.}
%      \entry {2020 -- actualidad} {ST} {\href{https://github.com/lonyelon/st}{\underline{\textbf{ver en Github}}}} {Mi propia versión del terminal de suckless.}
%      \entry {2020 -- actualidad} {DWM} {\href{https://github.com/lonyelon/dwm}{\underline{\textbf{ver en Github}}}} {Mi propia versión del gestor de ventanas de suckless.}
%      \entry {2020 -- actualidad} {DMENU} {\href{https://github.com/lonyelon/dmenu}{\underline{\textbf{ver en Github}}}} {Mi propia versión del menú de suckless.}
%      \entry {2020 -- actualidad} {Shell scripts} {} {Conjunto de scripts para automatización Github} {Un cliente de matrix escrito en Rust e inspirado en VIM.}
%    \end{entrylist}
%\end{minipage}

\cvsect{Tecnologías que domino}

Dejo una lista de tecnologías que conozco o he empleado en el pasado para facilitar la búsqueda:\\

\begin{tabular}{|c|c|}\hline

  Programación de bajo nivel &
  \texttt{C++}  \slashsep
  \texttt{C}    \slashsep
  \texttt{Rust} \\ \hline

  Programación de alto nivel &
  \texttt{Java}       \slashsep
  \texttt{Python}     \slashsep
  \texttt{Bash}       \slashsep
  \texttt{C\#}        \slashsep
  \texttt{JavaScript} \slashsep
  \texttt{PHP}        \\ \hline

  Servidores web &
  \texttt{Apache HTTPD} \slashsep
  \texttt{Nginx}        \slashsep
  \texttt{Caddy}        \\ \hline

  Bases de datos &
  \texttt{MySQL}      \slashsep
  \texttt{MariaDB}    \slashsep
  \texttt{PostgreSQL} \\ \hline

  Virtualización y contenedores &
  \texttt{Kubernetes} \slashsep
  \texttt{Openshift}  \slashsep
  \texttt{Docker}     \slashsep
  \texttt{Podman}     \slashsep
  \texttt{QEMU}       \slashsep
  \texttt{KVM}        \\ &
  \texttt{VirtualBox} \slashsep
  \texttt{VMWare}     \\ \hline

  Herramientas de diseño web &
  \texttt{Bootstrap}      \slashsep
  \texttt{Wordpress}      \slashsep
  \texttt{Nextcloud}      \\ \hline

  Herramientas de Devops  &
  \texttt{Ansible}        \slashsep
  \texttt{Git}            \slashsep
  \texttt{Apache Airflow} \slashsep
  \texttt{Jenkins}        \\ \hline

  Misceláneo &
  \texttt{Samba}   \slashsep
  \texttt{OpenMPI} \slashsep
  \texttt{CUDA}    \slashsep
  \texttt{SDL}     \slashsep
  \texttt{SFML}    \slashsep
  \texttt{NuxtJS}  \\ &
  \texttt{VueJS}   \slashsep
  \texttt{NodeJS}  \slashsep
  \texttt{Jira}    \slashsep
  \texttt{Hashicorp vault}  \\ \hline

\end{tabular}

\end{document}
