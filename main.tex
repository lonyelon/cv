% Sergio Miguens Iglesias' CV. This document is based on a template, see LICENSE
% for more details. You are free to copy and redistribute the code for my CV
% freely.

%                                      PACKAGES AND OTHER DOCUMENT CONFIGURATIONS
%%%%%%%%%%%%%%%%%%%%%%%%%%%%%%%%%%%%%%%%%%%%%%%%%%%%%%%%%%%%%%%%%%%%%%%%%%%%%%%%%

\documentclass[9pt]{developercv} % Default font size, values from 8-12pt are recommended
\usepackage{pagecolor,lipsum}
\usepackage{amssymb}
\usepackage{tikz}

% Cool little trick to have two languages in the same file.
\newif\ifen
\newif\ifes
\newcommand{\en}[1]{\ifen#1\fi}
\newcommand{\es}[1]{\ifes#1\fi}

% Set language to Spanish.
\estrue

\begin{document}

%                                                   TITLE AND CONTACT INFORMATION
%%%%%%%%%%%%%%%%%%%%%%%%%%%%%%%%%%%%%%%%%%%%%%%%%%%%%%%%%%%%%%%%%%%%%%%%%%%%%%%%%

% Name and title.
\begin{minipage}[t]{0.45\textwidth}
    \vspace{-\baselineskip}

    \colorbox{black}{{\HUGE\textcolor{white}{\textbf{\MakeUppercase{Sergio}}}}}
    \colorbox{black}{{\HUGE\textcolor{white}{\textbf{\MakeUppercase{M. Iglesias}}}}}

    \vspace{6pt}

    {\huge
      \es{Ingeniero informático}
      \en{Computer Engineer}
    }
\end{minipage}

% Photo in header.
\begin{minipage}[t]{0.45\textwidth}
  \vspace{-0.75cm}
  \hspace{5cm}
  \begin{tikzpicture}
    \clip (0,0) circle (1.5cm) node {\includegraphics[width=3cm]{img/me.jpg}};
  \end{tikzpicture}
\end{minipage}

\vspace{0.5cm}

%                                           INTRODUCTION, SKILLS AND TECHNOLOGIES
%%%%%%%%%%%%%%%%%%%%%%%%%%%%%%%%%%%%%%%%%%%%%%%%%%%%%%%%%%%%%%%%%%%%%%%%%%%%%%%%%

\begin{minipage}[t]{0.35\textwidth}
  \vspace{-\baselineskip} % Required for vertically aligning minipages
  \cvsect{
    \es{Sobre mí}
    \en{About me}
  }

  \es{Soy Sergio, un ingeniero informático de Vilagarcía de Arousa. Me apasionan los sistemas informáticos y la programación de bajo nivel. Soy usuario exclusivamente de Linux desde hace años.}
  \en{\lorem \lorem}
\end{minipage}
\begin{minipage}[t]{0.1\textwidth}
  \hspace{0.1cm}
\end{minipage}
\begin{minipage}[t]{0.275\textwidth} % 27.5% of the page width for the first row of icons
  \vspace{1cm}
  \vspace{-\baselineskip}

  % The first parameter is the FontAwesome icon name, the second is the box size and the third is the text
  % Other icons can be found by referring to fontawesome.pdf (supplied with the template) and using the word after \fa in the command for the icon you want
  \icon{MapMarker}{12}{Pontevedra}\\
  \icon{Phone}{12}{+34 671307714}\\
  \icon{At}{12}{\href{mailto:sergio@lony.xyz}{sergio@lony.xyz}}\\
\end{minipage}
\begin{minipage}[t]{0.275\textwidth} % 27.5% of the page width for the second row of icons
  \vspace{1cm}
  \vspace{-\baselineskip}

  % The first parameter is the FontAwesome icon name, the second is the box size and the third is the text
  % Other icons can be found by referring to fontawesome.pdf (supplied with the template) and using the word after \fa in the command for the icon you want
  \icon{Globe}{12}{\href{https://lony.xyz}{lony.xyz}}\\
  \icon{Github}{12}{\href{https://github.com/lonyelon}{github.com/lonyelon}}\\
  %\icon{Twitter}{12}{\href{https://twitter.com/@alyxvance}{@alyxvance}}\\
\end{minipage}

\cvsect{Habilidades}

\begin{minipage}[t]{1.0\textwidth}
  \begin{center}
    {\Large
      \begin{tabular}{rl}
        $\bigstar\bigstar\bigstar\bigstar\bigstar$ & \hspace{0.25cm} Programación de bajo nivel (C, C++, RUST), administración de sistemas,    \\
                                               & \hspace{0.25cm} trabajo en equipo, documentación, \ldots    \vspace{0.25cm} \\
        $\bigstar\bigstar\bigstar$                 & \hspace{0.25cm} Programación de alto nivel (Java, JS, Python), bases de datos, \ldots    \vspace{0.25cm} \\
        $\bigstar$                                 & \hspace{0.25cm} Diseño web, diseño de interfaces de usuario, \ldots   \vspace{0.25cm} \\
      \end{tabular}
    }
  \end{center}
\end{minipage}

%                                                                      EXPERIENCE
%%%%%%%%%%%%%%%%%%%%%%%%%%%%%%%%%%%%%%%%%%%%%%%%%%%%%%%%%%%%%%%%%%%%%%%%%%%%%%%%%

\cvsect{Estudios y experiencia}

\begin{entrylist}
  \entry {2021 -- actualidad} {Ingeniero de sistemas backup} {FDS -- a DXC Technology Company} {Mi trabajo consiste en administrar las copias de seguridad de miles de máquinas. Dada la naturaleza del trabajo, tengo que interactuar a dirario con sistemas Windows y Linux, hacer scripts y vivir en el terminal.

          \texttt{RHLE}\slashsep\texttt{Veritas Netbackup}\slashsep\texttt{Docker}\slashsep\texttt{Python}\slashsep\texttt{Apache Airflow}\slashsep\texttt{CheckMK}}
  \entry
      {2017 -- 2022}
      {Grado en ingeniería informática}
      {Escuela técnica superior de ingeniería, USC}
      {
        Si algo he aprendido del grado es a trabajar con una gran carga de trabajo, deadlines inminentes y equipos diversos

        Mi TFG fue un trabajo en equipo, en el que desarrollamos un entrenador personal virtual completo, incluyendo una página web, aplicación para Android, base de datos MySQL, CPD con Docker, y aplicación standalone para sistemas Windows. Este trabajoen equipo requirió de la aplicación de SCRUM y metodologías ágiles.
      }
\end{entrylist}

%                                                                 CERTIFICACIONES
%%%%%%%%%%%%%%%%%%%%%%%%%%%%%%%%%%%%%%%%%%%%%%%%%%%%%%%%%%%%%%%%%%%%%%%%%%%%%%%%%

\cvsect{Certificaciones oficiales}

\setlength{\tabcolsep}{12pt}
\begin{tabular}{cc}
  \textbf{DevOps Foundation} & \textbf{VMWare vSphere 6.7 Foundations} \\
  DevOps Institute & VMware \\
  \includegraphics[height=3cm]{img/cert_devops.jpg} & \includegraphics[height=3cm]{img/cert_vmware-foundations.png}
\end{tabular}\\

%                                                                          EXTRAS
%%%%%%%%%%%%%%%%%%%%%%%%%%%%%%%%%%%%%%%%%%%%%%%%%%%%%%%%%%%%%%%%%%%%%%%%%%%%%%%%%

\begin{minipage}[t]{0.3\textwidth}
  \vspace{-\baselineskip} % Required for vertically aligning minipages

  \cvsect{Idiomas}

  \begin{tabular}{rl}
  \textbf{Español}   & Nativo \\
  \textbf{Gallego}   & Nativo \\
  \textbf{Inglés}    & Competente \\
  \textbf{Portugués} & Básico
  \end{tabular}
\end{minipage}
\hfill
\begin{minipage}[t]{0.3\textwidth}
  \vspace{-\baselineskip} % Required for vertically aligning minipages

  \cvsect{Hobbies}

  Me encanta leer y la halterofilia. Me muero por el software libre y la filosofía UNIX.
\end{minipage}
\hfill
\begin{minipage}[t]{0.3\textwidth}
  \vspace{-\baselineskip} % Required for vertically aligning minipages

  \cvsect{Contribuciones a software libre}
    \begin{barchart}{5.5}
      \baritem{Kernel de Linux}{40}
      \baritem{Servidor de Matrix}{10}
    \end{barchart}
\end{minipage}

% TODO Complete this.
%\begin{minipage}[t]{0.9\textwidth}
%\cvsect{Proyectos personales}
%    \begin{entrylist}
%      \entry {2016 -- 2017} {CPU casera} {\href{https://www.reddit.com/r/computers/comments/65pxx3/8_bit_homebrew_computer/}{\underline{\textbf{ver en Reddit}}}} {Es una CPU de 8 bits que construí en bachillerato usando \textit{breadboards} y circuítos CMOS.}
%      \entry {2019} {Imitación de Minecraft en C++} {\href{https://github.com/lonyelon/Opencraft}{\underline{ver en Github}}} {Usa OpenGL y libnoise.}
%      \entry {2019} {Escáner de URLs} {\href{https://github.com/lonyelon/FuzzYourYelon}{\underline{ver en Github}}} {Se le proporciona una URL y la escanea en busca de ficheros.}
%      \entry {2020 -- actualidad} {ST} {\href{https://github.com/lonyelon/st}{\underline{\textbf{ver en Github}}}} {Mi propia versión del terminal de suckless.}
%      \entry {2020 -- actualidad} {DWM} {\href{https://github.com/lonyelon/dwm}{\underline{\textbf{ver en Github}}}} {Mi propia versión del gestor de ventanas de suckless.}
%      \entry {2020 -- actualidad} {DMENU} {\href{https://github.com/lonyelon/dmenu}{\underline{\textbf{ver en Github}}}} {Mi propia versión del menú de suckless.}
%      \entry {2020 -- actualidad} {Shell scripts} {} {Conjunto de scripts para automatización Github} {Un cliente de matrix escrito en Rust e inspirado en VIM.}
%    \end{entrylist}
%\end{minipage}

\cvsect{Tecnologías que domino}

Dejo una lista de tecnologías que conozco o he empleado en el pasado para facilitar la búsqueda:\\

\begin{tabular}{|c|c|}\hline

  Programación de bajo nivel &
  \texttt{C++}  \slashsep
  \texttt{C}    \slashsep
  \texttt{Rust} \\ \hline

  Programación de alto nivel &
  \texttt{Java}       \slashsep
  \texttt{Python}     \slashsep
  \texttt{Bash}       \slashsep
  \texttt{C\#}        \slashsep
  \texttt{JavaScript} \slashsep
  \texttt{PHP}        \\ \hline

  Servidores web &
  \texttt{Apache HTTPD} \slashsep
  \texttt{Nginx}        \slashsep
  \texttt{Caddy}        \\ \hline

  Bases de datos &
  \texttt{MySQL}      \slashsep
  \texttt{MariaDB}    \slashsep
  \texttt{PostgreSQL} \\ \hline

  Virtualización y contenedores &
  \texttt{Docker}     \slashsep
  \texttt{Podman}     \slashsep
  \texttt{QEMU}       \slashsep
  \texttt{KVM}        \slashsep
  \texttt{VirtualBox} \slashsep
  \texttt{VMWare}     \\ \hline

  Herramientas de diseño web &
  \texttt{Bootstrap}      \slashsep
  \texttt{Wordpress}      \slashsep
  \texttt{Nextcloud}      \\ \hline

  Herramientas de Devops  &
  \texttt{Git}            \slashsep
  \texttt{Apache Airflow} \slashsep
  \texttt{Jenkins}        \\ \hline

  Misceláneo &
  \texttt{Samba}   \slashsep
  \texttt{OpenMPI} \slashsep
  \texttt{CUDA}    \slashsep
  \texttt{SDL}     \slashsep
  \texttt{SFML}    \slashsep
  \texttt{NuxtJS}  \\ &
  \texttt{VueJS}   \slashsep
  \texttt{NodeJS}  \\ \hline

\end{tabular}

\end{document}
